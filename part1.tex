% первая часть

\section{Описание предметной области}
В ходе выполнения проекта, главным образом, решается задача преобразования двумерных изображений в трехмерные(GIF) на мобильных устройствах.

В последние годы заметное место в области преобразования и фильтрации изображений занимает задача преобразования двумерных изображений в трехмерные. На сегодняшний день в мире для этого разработаны различные методики, которые позволяют автоматически создавать так называемые «карты глубины» для двумерных изображений, основываясь на свойствах этого изображение и на некоторых предположениях о характере сцены. 

\subsection{Android Studio}
Android Studio - полностью укомплектованная платформа для разработки и тестирования приложений под операционную систему Android. Разработчики этой оболочки (компания Google) внедрили весь необходимый инструментарий для удобного и качественного проектирования новых приложений и доработки существующих. Программа включает в себя такие компоненты как Android SDK, все версии операционки Android, эмулятор для запуска приложений, элементы тестирования и отладки программ.

Создавая новый проект, будет доступна полная структура приложения со всеми файлами, что позволяет более четко и продуманно организовать сам процесс разработки. Очень удобно реализован показ вносимых изменений и дополнений - визуально в реальном времени происходят преобразования в зависимости от заданных действий. Что немаловажно, программа позволяет делать разработку приложений для всех версий ОС Andriod и для различных устройств - можно предварительно оценить внешний вид программы, например, под планшет или смартфон.

Среда Android Studio предназначена как для небольших команд разработчиков мобильных приложений (даже в количестве одного человека), или же крупных международных организаций с GIT или другими подобными системами управления версиями. Опытные разработчики смогут выбрать инструменты, которые больше подходят для масштабных проектов. Решения для Android разрабатываются в Android Studio с использованием Java или C++. В основе рабочего процесса Android Studio заложен концепт непрерывной интеграции, позволяющий сразу же обнаруживать имеющиеся проблемы. Продолжительная проверка кода обеспечивает возможность эффективной обратной связи с разработчиками. Такая опция позволяет быстрее опубликовать версию мобильного приложения в Google Play App Store. В Android Studio есть удобная маркировка кода, которая позволит без труда ориентироваться в больших проектах. Кроме того, отдельные компоненты можно изменять простым перетаскиванием в другое нужное место, что значительно упрощает редактирование.

\subsection{Java}
Преимущества языка Java
\begin{itemize}
	\item Одно из основных преимуществ языка Java — независимость от платформы, на которой выполняются программы: один и тот же код можно запускать под управлением операционных систем Windows, Solaris, Linux, Machintosh и др. 
	Это действительно необходимо, когда программы загружаются через Интернет для последующего выполнения под управлением разных операционных систем.
	\item Другое преимущество заключается в том, что синтаксис языка Java похож на синтаксис языка C++, и программистам, знающим языки С и C++, его изучение не составляет труда.
	\item Кроме того, Java — полностью объектно-ориентированный язык, даже в большей степени, чем C++. Все сущности в языке Java являются объектами, за исключением немногих основных типов (primitive types), например чисел.
	\item Исключена возможность явного выделения и освобождения памяти.	Память в языке Java освобождается автоматически с помощью механизма сборки мусора. Программист гарантирован от ошибок, связанных с неправильным использованием памяти.
	\item Безопасный: методы проверки подлинности основаны на шифровании с открытым ключом.
	\item Динамический: программирование на Java считается более динамичным, чем на C или C++, так как он предназначен для адаптации к меняющимся условиям. Программы могут выполнять обширное количество во время обработки информации, которая может быть использована для проверки и разрешения доступа к объектам на время выполнения.
\end{itemize}

\subsection{Выбор архитектуры мобильного приложения}
В ходе выполнения работы были рассмотрены различные варианты для создания мобильных приложений, предназначенных для преобразования 2D изображений в 3D вид. При этом рассмотрении учитывалось, что результирующие мобильные приложения должны создаваться под операционные системы Android, а также то, что один из основных результатов работы приложения с точки зрения конечного пользователя – это возможность публикации созданного 3D-изображения в виде анимированного gif-файла в одном или нескольких аккаунтов в социальных сетях пользователя. Соответственно, можно исходить из предположения о том, что для функционирования приложения в любом случае необходим доступ к сети интернет. Максимальная унификация различных составных частей приложения между собой хотя бы на уровне исходных кодов вне зависимости от целевой платформы (Android или iOS) является дополнительным преимуществом при рассмотрении различных вариантов создания мобильных приложений.

Один из наиболее простых с технической точки зрения вариантов реализации решения, позволяющего преобразовывать 2D файлы в 3D вид, является решение, основанное на создании веб-сервиса, который предоставляет минимально необходимый пользовательский интерфейс для загрузки желаемого файла на сервер, преобразования файла на сервере и, как результат, возможность скачать получившийся файл на устройство пользователя и поделиться этим файлом в социальных сетях. При простоте архитектуры у этого решения есть один существенных недостаток – как правило, такие решения менее удобны и функциональны, чем нативные (native) мобильные приложения, разработанные специально под целевую платформу, на которой они будут функционировать.

Рассмотрим два варианта создания нативных мобильных приложений:

\begin{enumerate}
	\item Использовать наиболее популярные средства разработки и языки программирования для каждой из необходимых мобильных платформ. Создать нативное мобильное приложение, реализующее весь необходимый пользовательский интерфейс, набор сервисных функций. Портировать алгоритм преобразования графического файла из 2D в 3D для локального исполнения на мобильном устройстве. Все необходимые преобразования выполнять локально, на мобильном устройстве. Полученный результат преобразования (анимированный gif) загружать в интернет (социальные сети) по мере его готовности на мобильном устройстве.
	
	\item Использовать наиболее популярные средства разработки и языки программирования для каждой из необходимых мобильных платформ для создания нативных мобильных приложений только для реализации пользовательского интерфейса и набора сервисных функций. Алгоритм преобразования графического файла из 2D в 3D реализуется в виде серверного модуля, соответственно для преобразования выбранного файла и предварительного просмотра полученных результатов необходимо загрузить этот выбранный файл на сервер. Загрузить полученный результат с сервера и поделиться этим результатом в социальных сетях.
\end{enumerate}

Для варианта №1 для операционной системы Android необходимо:

С использованием Android Studio на языке программирования Java реализовать необходимый пользовательский интерфейс, а также весь необходимый набор сервисных функций. Необходимо адаптировать реализацию алгоритма преобразования графического файла из 2D в 3D для использования под управлением операционной системы Android (реализация на С++). Далее, с использованием механизма The Android Native Development Kit (NDK) необходимо обеспечить вызов кода, написанного на языке С++ из «классического» Android-приложения. 

Для реализации варианта №2 необходимо:

С использованием Android Studio на языке программирования Java необходимо создать нативное мобильное приложение для реализации пользовательского интерфейса и набора сервисных функций. Эта задача, в целом, является типовой и принципиальных сложностей не вызывает. Алгоритм преобразования графического файла из 2D в 3D следует реализовать в виде серверного модуля, например, для использования под управлением операционной системы Ubuntu. Это обусловлено тем, что Unix-подобные операционные системы имеют существенно более широкое распространение в Web-серверном окружении, чем Windows-сервера.

На основе проведенного исследования моно сделать следующий вывод. С точки зрения скорости, легкости и качества реализации наиболее перспективными являются вариант №2. 

Очевидным недостатком подобного решения является существенная его зависимость от скорости и надежности мобильного интернета, а также от доступности конечному пользователю оплаченного траффика. Для обхода этих ограничений предполагается исследовать возможность создания для пользователей ОС Android «самодостаточного» мобильного приложения (вариант №1), которое все необходимые действия, связанные с преобразованием файлов производит непосредственно на мобильном устройстве.

\section{Изучение основ разработки мобильного приложения в среде Android Studio}

Android основан на Linux. Между приложением и ядром лежит слой API и слой библиотек на нативном коде. Приложение выполняется на виртуальной машине Java (Dalvik Virtual Machine).
В Android можно запускать много приложений. Но одно из них есть главным и занимает экран. От текущего приложения можно перейти к предыдущему или запустить новое. Это похоже на браузер с историей просмотров.

Каждый экран пользовательского интерфейса представлен классом Activity в коде. Различные Activity содержатся в процессах. Activity может даже жить дольше процесса. Activity может быть приостановлена и запущена вновь с сохранением всей нужной информации.(рисунок~\ref{fig:activity})

\begin{figure}[H]
	\centering
	\includegraphics[width=0.6\linewidth]{pics/activity}
	\caption{activity}
	\label{fig:activity}
\end{figure}

Android использует специальный механизм описания действий основанный на Intent. Когда нужно выполнить действие (сделать звонок, послать письмо, показать окно), вызывается Intent.

Также Android содержит сервисы подобные демонам в Linux для выполнения нужных действий в фоновом режиме (например, проигрывание музыки).
Для обмена данными между приложениями используются Content providers (провайдеры содержимого).

Содержимое Activity формируется из различных компонентов, называемых View. Самые распространенные View - это кнопка, поле ввода, чекбокс и т.д. (рисунок~\ref{fig:view})

\begin{figure}[H]
	\centering
	\includegraphics[width=0.6\linewidth]{pics/view}
	\caption{view}
	\label{fig:view}
\end{figure}

Необходимо заметить, что View обычно размещаются в ViewGroup. Самый распространенный пример ViewGroup – это Layout. Layout бывает различных типов и отвечает за то, как будут расположены его дочерние View на экране.

LinearLayout – отображает View-элементы в виде одной строки (если он Horizontal) или одного столбца (если он Vertical).

TableLayout – отображает элементы в виде таблицы, по строкам и столбцам.

RelativeLayout – для каждого элемента настраивается его положение относительно других элементов.

AbsoluteLayout – для каждого элемента указывается явная позиция на экране в системе координат (x,y)

\section{Основные компоненты пользовательского интерфейса мобильного приложения в Android}

В Android используется UI-фреймворк, сравнимый с другими полнофункциональными UI-фреймворками, применяемыми на локальных компьютерах. Он является более современным и асинхронным по природе. По существу, UI-фреймворк Android относится уже к четвертому поколению, если считать первым поколением традиционный прикладной интерфейс программирования Microsoft Windows, основанный на С, а MFC (Microsoft Foundation Classes, библиотека базовых классов Microsoft на основе C++) - вторым. В таком случае UI-фреймворк Swing, основанный на Java, будет третьим поколением, так как предлагаемые в нем возможности дизайна значительно превосходят по гибкости MFC. Android UI, JavaFX, Microsoft Silverlight и язык пользовательских интерфейсов Mozilla XML (XUL) относятся к новому типу UI-фреймворков четвертого поколения, в котором UI является декларативным и поддерживает независимую темизацию.

При программировании в пользовательском интерфейсе Android применяется объявление интерфейса в файлах XML. Затем эти определения представления (view definitions) XML загружаются в приложение с пользовательским интерфейсом как окна. Даже меню приложения загружаются из файлов XML. Экраны (окна) Android часто называются активностями (activities), которые включают в себя несколько видов, нужных пользователю, чтобы выполнить логический элемент процесса. Виды (views) являются основными элементами, из которых в Android состоит пользовательский интерфейс. Виды можно объединять в группы (view groups). Для внутренней организации видов используются давно известные в программировании концепции холст (canvas), рисование (painting) и взаимодействие пользователя с системой (user interaction).

Такие составные представления, в которые входят виды и группы видов, работают на базе специального логического заменяемого компонента пользовательского интерфейса Android.

Одной из ключевых концепций фреймворка Android является управление жизненным циклом (lifecycle) окон явлений (activity windows). В системе применяются протоколы, поэтому Android может управлять ситуацией по мере того, как пользователи скрывают, восстанавливают, останавливают и закрывают окна явлений.


\section{Проектирование структуры приложения}

Одна из важнейших задач в ходе создания мобильного приложения, преобразовывающего 2D снимки в объемные (3D) - разработка удобного и интуитивно-понятного пользовательского интерфейса. UI/UX (user Interface, user experience) составляющая, она же пользовательский интерфейс и пользовательский опыт, является более чем просто значимым элементом современного мобильного приложения. Удобство расположения элементов управления и приятное визуальное оформление напрямую влияют на настроение пользователя при использовании продукты. Именно некачественный UI/UX-дизайн отпугивает людей от использования многих приложений в пользу их более достойных альтернатив.

На главный экран камеры (рисунок~\ref{fig:Artboard}) выведены следующие функции:

\begin{itemize}
	\item Спуск затвора;
	\item Выбор фото из галереи;
	\item Смена камеры;
	\item Управление вспышкой;
	\item Переход к настройкам и справке.
\end{itemize}

\begin{figure}[H]
	\centering
	\includegraphics[width=0.6\linewidth]{pics/Artboard}
	\caption{Окно с камерой}
	\label{fig:Artboard}
\end{figure}

На экране с обработанной фотографией по нажатии кнопки «Далее» появляется bottom sheet (рисунок~\ref{fig:Artboard2}), включающий в себя быстрые функции шеринга и сохранения полученного фото в галерею.

\begin{figure}[H]
	\centering
	\includegraphics[width=0.6\linewidth]{pics/Artboard2}
	\caption{Окно с просмотром фото}
	\label{fig:Artboard2}
\end{figure}

При создании пользовательского интерфейса приложения были проанализированы современные, с аналогичным функционалом мобильные приложения в целом. Был сделан акцент на необходимости создания современного, функционального и не перегруженного пользовательского интерфейса. В результате проведенного анализа был создан пользовательский интерфейс мобильного приложения для преобразования 2D фотографий в 3D вид.

\section{Разработка мобильного приложения}

Рассмотрю практичный пример, когда программно запускаю приложение "Камера", а полученную фотографию сохраняю в папке.~\cite{camera}

В манифесте нужно добавить разрешение на запись файла в хранилище и указать требование наличия камеры.

Используем статическую константу ACTION\_IMAGE\_CAPTURE из объекта MediaStore для создания намерения, которое потом нужно передать методу startActivityForResult(). Разместим на форме кнопку и ImageView, в который будем помещать полученный снимок. Полученное с камеры изображение можно обработать в методе onActivityResult()

При тестировании примера на своём телефоне я обнаружил небольшую проблему - когда снимок передавался обратно на моё приложение, то оно находилось в альбомном режиме, а потом возвращалось в портретный режим. При этом полученный снимок терялся. Поэтому перед нажатием кнопки я поворачивал телефон в альбомный режим, чтобы пример работал корректно. Поэтому надо предусмотреть подобное поведение, например, запретить приложению реагировать на поворот и таким образом избежать перезапуска Activity. 

По умолчанию фотография возвращается в виде объекта Bitmap, содержащего миниатюру. Этот объект находится в параметре data, передаваемом в метод onActivityResult(). Чтобы получить миниатюру в виде объекта Bitmap, нужно вызвать метод getParcelableExtra() из намерения, передав ему строковое значение data.

(рисунок~\ref{fig:my})

\begin{figure}[H]
	\centering
	\includegraphics[width=0.4\linewidth]{pics/main}
	\includegraphics[width=0.4\linewidth]{pics/camera}
	\caption{Результат работы приложения}
	\label{fig:my}
\end{figure}